\chapter{Introduction}

Code clones are fragments of code, in a single or multiple programs, which are
similar to each other. Code clones can emerge for various reasons, such as
copy-pasting or implementing the same functionality as another developer. Code
clones can decrease the maintainability of a program, as it becomes necessary to
fix an error in all the places where the code was copied. However, detecting
these code clones --- which is often referred to as the task of code clone
detection --- is a difficult task and has been extensively researched in the
literature. Some systems focus on finding clones inside a single project, while
other task try to detect clones in larger ecosystem, by for example trying to
detect clones in open-source development websites such as GitHub.
With the variety of tools and approaches proposed for clone detection, the task
of code clone detection has also found various applications, such as helping for
code search, malware detection, copyright infringement detection, and many
others. However, although there is a very large amount of tools that have been
developed for the task of clone detection, most of these have been developed to
detect clones in programs written in the same programming language, and the task
of detecting code clones for programs written in different languages has not
been studied as well in the literature.

In recent years, the number of programming languages typically used by a
developer has increased~\cite{programming-languages-over-time}, and developers
tend to switch between programming languages much more often than a decade ago.
There are many possible explanations to this phenomenon. One reason is that many
companies use different languages for prototyping and production, and
programmers therefore tend to use a scripting language such as Python to develop
a prototype and a faster language such as C++ for the production code. Another
explanation is that native applications often need to be written in different
programming languages: iOS applications can be written in Swift or Objective C,
Android applications require a language running on the JVM, such as Java or
Kotlin, and application running in the browser need to be written in JavaScript.
Yet another potential reason for this proliferation of languages is the
democratization of micro services. As multiple parts of a single system often
run in total isolation from each other in their own process or container,
the need to write everything in a single language is not as high as it used to
be with monolithic applications and teams therefore tend to choose the
programming language the most adapted for a given task, and having a system with
a websocket server running JavaScript code and another server performing
machine learning tasks using Python is not rare.

Given such a context, where it is common for a system to be implemented using
different programming languages, code clones between part of the systems
implemented in different programming languages will eventually emerge, causing
the same kind of maintainability issues as code clones in a program written in
the same programming language. To help improving this situation, the question of
whether we can detect code clones in programs written in different programming
languages is therefore important.

In this thesis, our goal is to find a method to detect code clones written in
different programming languages. Given the variety of languages and the
potential difference between them, rule-based comparison approaches seem
challenging to apply in this context. We therefore decided to focus on machine
learning based techniques to perform this task. We design a system to perform
this task, collect data to train the system, and evaluate our system to see how
well it performs. In this thesis, we present two main contributions.

\begin{itemize}
\item We present a system based on supervised machine learning capable of
  detecting clones across programming languages. We train and evaluate our model
  using programs written in Python and Java, and obtain promising results in
  terms of precision and recall. Our system being language agnostic, the only
  requirement for being able to support new programming languages is to be able
  to collect code clones data written in the desired language.
\item We present a cross-language code clones dataset containing about 50000
  files written in Java and Python. Although code clones dataset exist for
  programming languages such as Java, this is to the best of our knowledge the
  first dataset with cross-language code clones. We collected the data for this
  dataset by scraping competitive programming websites and used the data
  provided by the websites to extract code clones.
\end{itemize}

We also present in detail how we transform the programs to be able to feed them
to a machine learning model and show how applying natural language processing
methods can improve our results.

The rest of this thesis is organized as follow. In
Chapter~\ref{ch:background}, we give some background about the task we are
trying to solve by presenting the necessary concepts, giving motivating examples
and discussing the related work in this area. In Chapter~\ref{ch:proposal}, we
give details about our proposal, we describe the different parts of our system
and explain how our machine learning model works and how we train it. In
Chapter~\ref{ch:experiments}, we describe the different experiments we
performed, show and discuss the results we obtained. Finally, in
Chapter~\ref{ch:conclusion}, we summarize what we have presented in this thesis
and give some directions for future work in this area.
