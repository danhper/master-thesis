\begin{eabstract}
While clone detection across programs written in the same programming language
has been studied extensively in the literature, the task of detecting clones
across multiple programming languages is not covered as well, and approaches
based on comparison cannot be directly applied.
In this thesis, we present a clone detection method based on supervised
machine learning able to detect clone across programming languages.
Our method uses an unsupervised learning approach to learn token-level vector
representations and an LSTM-based neural network to predict if two code
fragments are clones. To train our network, we present a cross-language code
clone dataset --- which is to the best of our knowledge the first of its kind
--- containing more than 50000 code fragments written in Python and Java.
We show that our method can detect code clones between Python and Java with high
accuracy. We also compare our method to state-of-the-art tools in
single-language clone detection and show that we achieve similar accuracy.
\end{eabstract}

\begin{jabstract}
クローン検出は複数のプログラムから同じようなソースコードを探し出すタスクである。\\
コード検索、リファクタリング、著作権違反検出などのような多くの応用を持っており、%
マルウェア検出などの基礎技術としても利用されている。同一の言語でのクローン検出が%
よく研究されているものの、複数の言語間でのクローン検出を行う研究がまだ少ない。%
一方、複数のプログラミングを使うことが一般的になってきている。1つのプロジェクトでも、%
メージャーなプラットフォームのネィティブアプリケーションを持つだけで%
プログラミング言語を3つ使用することになる。よって、言語間でのクローン検出も%
重要になってきている。

本研究では、教師あり機械学習を用いた言語間でのクローンを検出する方法を提案する。\\
まず、ソースコードを機械学習のアルゴリズムで使えるようにするための前処理の手順%
を説明する。続いて、言語間でのクローン検出が行えるモデルを紹介する。このモデル%
に与えるために作成したJavaとPythonの約5万ファイルのデータセットについて説明する。%
最後に、提案手法を用いて高い精度でクローン検出が行えたことを示す。
\end{jabstract}
