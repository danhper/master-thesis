\chapter{\label{ch:conclusion}Conclusion}
\section{Summary}
In this thesis, we presented a system capable to detect code clones across
programs written in different programming languages.

We gave some background about the motivations behind cross-language clone
detection, and explained the difficulties that can be encountered when trying to
achieve this task. In particular, we gave some explanation about why many
approaches used for clone detection in a single programming languages cannot be
applied directly, and why machine learning seems to be a good fit for this task.

We then presented a supervised learning based machine approach to detect
code clones. We introduced an LSTM based model which predicts if two programs
are clones or not, we gave details about how we compute token embedding for
a particular programming language, and showed how we can pre-process source code
and use these embedding to feed data to our model.

We introduced a novel dataset for cross-language code clone
detection composed of programs written in Java and Python, containing around
50000 files and implementing more than 500 different functionalities. We
explained the motivation behind the choice of competitive programming to collect
programs for this dataset, and gave details about the methodology used to
generate the dataset.

We then showed the results we obtained when evaluating our trained system using
our dataset. We demonstrated that we are able to obtain a high recall, with
scores around $0.7$, while keeping a reasonable precision --- with scores in the
order of $0.3$ to $0.5$ --- given the motivation of our system. We also
presented the results we obtained when training token-level embedding using a
large set of data we fetched from GitHub, and described how it helps us achieve
better results for our clone detection task.
\section{Future work}
Although we obtained some promising results for the task of cross-language code
clone detection in terms of recall and precision, scalability is a major issue
of our current approach. In the current implementation of our system, our model
takes as input two code fragments, and determines if they are code clones or
not. This works correctly when evaluating our model by feeding it pairs of
clones directly. However, in a real-world use case, this means that the clone
detection process would take $\mathcal{O}\left(n^2\right)$, where $n$ is the
number of code fragments to evaluate. As code fragments are most commonly code
blocks, and the number can therefore be in the order of $100$ millions in a
large enough project, our method as is would not work. We should be able to
modify our system to add a hash-layer as done for example
in~\cite{ijcai2017-423} and use a method such as locality-sensitive
hashing~\cite{Datar:2004:LHS:997817.997857} to reduce the order of our system to
$\mathcal{O}(n)$, making it suitable for real-world applications.
